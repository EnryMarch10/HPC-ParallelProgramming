\documentclass[12pt, a4paper]{article}

\usepackage[margin=1in]{geometry}
\usepackage[utf8]{inputenc}
\usepackage[T1]{fontenc}
\usepackage[italian]{babel} % Set language to Italian

\usepackage{times}

\usepackage{float}

\usepackage{url}
\usepackage{xurl} % Avoids URLs to overfull \hbox

\usepackage{graphicx}
% \graphicspath{{img/}} % global configuration
\usepackage[colorlinks=false, pdfborder={0 0 0}]{hyperref}

\usepackage{tabularray}

% To use listings
\usepackage[table, svgnames]{xcolor}
\usepackage{listings}

\title{
  Progetto di High Performance Computing 2024/2025
}
\author{
  Enrico Marchionni, matr. 0001032322
}
\date{\today}

% Package to keep track of the total number of pages
\usepackage{lastpage}
\usepackage{fancyhdr}

\fancypagestyle{fancy}{
  \fancyhf{}
  \fancyfoot[C]{\thepage\ of \pageref{LastPage}}
  \renewcommand{\headrulewidth}{0pt}
  \renewcommand{\footrulewidth}{0.4pt}
}

\fancypagestyle{plain}{
  \fancyhf{}
  \fancyfoot[C]{\thepage\ of \pageref{LastPage}}
  \renewcommand{\headrulewidth}{0pt}
  \renewcommand{\footrulewidth}{0.4pt}
}

\pagestyle{fancy}

\usepackage{amsthm} % to make definitions
\newtheorem{definition}{Definition}[section] % custom environment for definitions
\newtheorem{example}{Example}

% for graphs and overlaying text
\usepackage{tikz}
\usepackage{pgfplots}
\pgfplotsset{compat=1.18}
\usepackage{amsmath} % For math symbols

\definecolor{C_comment_green}{rgb}{87, 166, 74}
\definecolor{C_types_blue}{RGB}{30, 130, 210}
\definecolor{C_user_types_green}{RGB}{50, 222, 156}
\definecolor{C_directives_purple}{RGB}{140, 10, 133}
\definecolor{C_function_amber}{RGB}{200, 180, 20}

\lstdefinelanguage{CStyle}{
  morekeywords={ % types
    int, const, float
  },
  morekeywords=[2]{ % user types
    points_t
  },
  morekeywords=[3]{ % directives
    pragma, for, if, return
  },
  morekeywords=[4]{ % functions
    skyline, dominates
  },
  sensitive=true,
  morecomment=[l]//,
  morecomment=[s]{/*}{*/},
  morestring=[b]",
  morestring=[b]',
  commentstyle=\color{C_comment_green},
  keywordstyle=\color{C_types_blue}\bfseries,
  keywordstyle=[2]\color{C_user_types_green}\bfseries,
  keywordstyle=[3]\color{C_directives_purple}\bfseries,
  keywordstyle=[4]\color{C_function_amber}\bfseries,
  stringstyle=\color{red},
  identifierstyle=\color{darkgray},
  basicstyle=\ttfamily,
  literate={->}{{\textcolor{black}{->}}}2
           {>=}{{\textcolor{black}{>=}}}2
           {<=}{{\textcolor{black}{<=}}}2
           {!=}{{\textcolor{black}{!=}}}2
           {==}{{\textcolor{black}{==}}}2
           {(}{{\textcolor{black}{(}}}1
           {)}{{\textcolor{black}{)}}}1
           {\{}{{\textcolor{black}{\{}}}1
           {\}}{{\textcolor{black}{\}}}}1
}

\lstset{
  basicstyle=\ttfamily,                 % Set font type
  keywordstyle=\color{blue}\bfseries,   % Set keyword color to blue and bold
  stringstyle=\color{red},              % Set string color to red
  commentstyle=\color{C_comment_green}, % Set comment color to gray and italic
  breaklines=true,                      % Enable line breaking
  frame=single,                         % Add a frame around the code
  numbers=left,                         % Add line numbers on the left
  numberstyle=\tiny\color{gray},        % Style for line numbers
  backgroundcolor=\color{lightgray!20}, % Background color
  captionpos=b,                         % Caption position at the bottom
  showstringspaces=false                % Don't show spaces in strings
}

\begin{document}

\maketitle

\section{Introduzione}

\dots

\section{Versione Seriale}

L'algoritmo implementato è:

\begin{lstlisting}[language=CStyle, caption={Algoritmo per il calcolo dello skyline in C}, label={lst:algoritmo_skyline_seriale}]
int skyline(const points_t *points, int *s)
{
  const int D = points->D;
  const int N = points->N;
  const float *P = points->P;
  int r = N;

  for (int i = 0; i < N; i++) {
    s[i] = 1;
  }

  for (int i = 0; i < N; i++) {
    if (s[i]) {
      for (int j = 0; j < N; j++) {
        if (s[j] && dominates(&(P[i * D]), &(P[j * D]), D)) {
          s[j] = 0;
          r--;
        }
      }
    }
  }
  return r;
}
\end{lstlisting}

È l'algoritmo seriale proposto è tipico per soluzioni brute-force del problema dello skyline.
Il costo computazionale dell'algoritmo in \autoref{lst:algoritmo_skyline_seriale} è in generale \(O(N^2D)\).
Infatti il ciclo esterno viene percorso \(N\) volte, nelle quali, nel caso peggiore, vengono eseguite altrettante \(N\) operazioni
di confronto tra \(D\) elementi (da cui \(N \times N \times D\)).
Si può notare che \(D\) può influire anche molto nella complessità, in particolare se \(D >> N^2\).
Nel caso in cui \(N^2 >> D\), si considera \(O(N^2)\), nel caso in cui \(D >> N^2\), \(O(D)\), nell'ultimo caso in cui
\(D \approx N^2\) allora \(O(N^2D)\).

\begin{table}[H]
  \begin{tblr}{
      colspec={*{3}{X[l]}},
      width=\textwidth,
      row{odd}={gray!15},
      row{even}={white},
      row{1}={bg=gray!90,fg=white},
      colsep=4pt
    }
      \textbf{Caso} & \textbf{Complessità} & \textbf{Range} \\
      Pessimo & \(\Theta(N^2D)\) & Tutti i punti sono dello skyline. \\
      \hline
      Medio & \(\Theta(N^2D)\) & La metà dei punti fanno parte dello skyline. \\
      \hline
      Ottimo & \(\Theta(ND)\) & Il primo punto analizzato domina tutti gli altri. \\
      \hline
  \end{tblr}
  \caption{\label{tab:algoritmo_skyline_complessita_casi} Input data.}
\end{table}

Come si può vedere dalla \autoref{tab:algoritmo_skyline_complessita_casi} la complessità in realtà varia anche di molto in base
all'input fornito. Fattore importante nella misura della weak scaling efficiency per esempio.

\section{Versione OpenMP}

\dots

\section{Versione MPI/CUDA}

\dots

\section{Conclusioni}

\dots

\end{document}
